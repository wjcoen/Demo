
\section{Introduction}\label{sec:intro}

Repurchase agreements (repos) are the main source of wholesale funding for financial institutions. As a result, repo markets plays a key role in shaping financial crises, market liquidity, and the transmission of monetary policy. A repo transaction serves two functions: firms may trade to obtain funding (liquidity demand), but they may also trade to temporarily obtain the asset provided as collateral (collateral demand). The funding dimension of repo is relatively well understood.\footnote{See for example \cite{julliard2019drives} and \cite{brunnermeier2009market}.} The empirical importance of collateral demand, and how it interacts with funding demand, is less well understood. How does collateral demand vary in the cross-section and in the time series? How does it interact with funding demand? What is the contribution of collateral demand to repo market functioning?

We answer these questions based on the Sterling gilt repo market, where we have transaction data on close to the universe of repo lending and borrowing backed by UK government bonds, from January 2017 to March 2023. We use these data to establish new facts on the role of demand for collateral, and to build and structurally estimate a model of the repo market. We use our estimated model to quantify the contribution of collateral demand to trade quantities, interest rates and firm payoffs. We find that removing collateral demand would significantly increase the cost of repo funding, decrease repo activity and decrease welfare, demonstrating that collateral demand is key to understanding repo markets.

We begin by describing three empirical facts in support of collateral demand as a key motive for trading in the Sterling repo market. First, lending banks frequently charge a lower rate than the risk-free rate. Second, hedge funds charge lower interest rates when lending than money market funds, whose limited mandates preclude a motive to demand specific gilts as collateral, and these rates are more sensitive to the precise gilt chosen as collateral. Third, interest rates are higher when the lender does not specify exactly which gilt it requires as collateral.

These facts are difficult to rationalise if the collateral is valued only as insurance against risk, and instead suggest that certain traders have collateral demand for specific gilts and are willing to lend at lower rates in return for being provided these gilts. This naturally raises further questions about (1) the scale of this collateral demand motive, and the extent to which it affects equilibrium trade and (2) what else could vary across transactions and confound these effects, including firms' unobserved funding needs, their position within the trading network or the funding needs of their other counterparties. To help with both of these we build and estimate a model of repo trading.

In the model, repos are a temporary exchange of cash for an asset. Firms use the cash they obtain to fund a risky project, but also use any assets they obtain as collateral to obtain a risky return (from shorting the asset, for example). There are multiple assets, representing each of the gilts, and the firms simultaneously write repos against any of these assets. There are two types of firms, dealers and customers, connected by an exogenous trading network that governs the set of customers with which each dealer can trade. Dealers also have access to a competitive inter-dealer market. Firms have mean-variance preferences and dealers have market power in their transactions with customers. Beyond their type and position in the network, firms are heterogeneous in the expected return they earn from cash (their funding demand) and the expected return they earn from each of the different types of collateral (their collateral demand). 


%There is an exogenous trading network that governs the set of customers with which each dealer can trade, and dealers also have access to a competitive inter-dealer market. Dealers and customers can use the cash they obtain on the repo market to fund a risky project, but they can also use the asset that they temporarily obtain as collateral to obtain a risky return. Firms are heterogeneous in the expected return they earn from cash (their funding demand) and the expected return they earn from each of the different types of collateral (their collateral demand). Firms have mean-variance preferences and dealers have market power in their transactions with customers.

The model pins down a unique equilibrium in which a firm's portfolio choices -- its repo borrowing and lending against all assets -- depend on its demands for cash and collateral, and those of its counterparties. Firms with low funding demand and high collateral demand are natural lenders. Dealers intermediate between natural lenders and borrowers and earn a spread, but also have their own demand for cash and collateral. The effect of collateral demand on market outcomes depends on the size of the collateral demand motive, but also on its joint distribution with funding demand: if collateral demand is negatively correlated with funding demand, then collateral demand increases trading volumes as it provides an additional motive for lenders to lend. If the two are positively correlated, then the reverse is true and collateral demand can reduce trading volumes.

We then estimate our model. Our objective is to use our model and the transaction data to recover the joint distribution of funding demand and collateral demand across time and across firms as flexibly as possible. Our model gives us equilibrium net demand curves, which govern how a given firm's net lending varies with the interest rate and other lending choices across the rest of the network. We estimate these net demand curves in two steps.  First, our model implies that a given dealer charges its counterparties different interest rates depending on risk aversion, volatility and the size of the trade. We estimate this model-implied regression of interest rates to recover risk aversion and volatility. We include firm-gilt-time fixed effects and as an instrument for trading quantity we use shocks to the prices of the bonds commonly used as collateral by firm $j$ to trace out the net demand of firm $i$.

In the second step to our estimation, we use these estimated fixed effects and parameters to back out the gilt-firm-time-specific intercepts of the net demand functions. This is analogous to stripping out the effect of network position and counterparty demand on interest rates, and leaving only a given firm's funding and collateral demand. Importantly, funding demand is common across assets (a firm's demand for cash does not depend on the underlying collateral), whereas collateral demand is specific to assets. Thus using variation in interest rates within firm-time but across gilts we can separately identify collateral and funding demand whilst making relatively few assumptions about their joint distribution.

Our estimates give us variation in the demand for funding across firms and time, and for collateral across firms, time and gilts. The time path of repo funding demand is closely related to the UK's monetary policy stance. As monetary policy tightened from 2022 onwards, the marginal cost of repo funding rose too. Collateral demand, by contrast, rises during times of stress in financial markets. In particular, collateral demand spiked during the dash-for-cash in March 2020, and during the gilt market turmoil in autumn 2022. This suggests that collateral demand is highest during times of stress, consistent with demand for short selling being a key driver of repo trading.

We quantitatively study the role of collateral demand in a counterfactual analysis in which we set the parameters governing collateral demand equal to zero. We find that, relative to our baseline estimated collateral demand parameters, this increases the average interest rate by 16\%, as it removes an important motivation for lenders to lend. This causes overall trading volumes to shrink by 70\% and aggregate firm utilities to decrease by over half. These results highlight the importance of collateral demand for repo markets. Collateral demand has a major impact on repo markets, and there are significant complementarities between collateral demand and the ability of firms’ to access cheap funding in large quantity via the repo market. Treating repo markets purely as funding markets, and ignoring the collateral-driven motive for trade, thus omits something of fundamental importance to repo markets.

Our work suggests that collateral demand is of first-order importance in shaping how repo markets function. Collateral demand varies across firms, assets and time, and the nature of collateral demand determines the quantity and price of funding that firms can obtain via repo. 

In Section \ref{sec:instSettingData}, we describe the relevant institutional detail of the UK repo market and our data. In Section \ref{sec:empiricalFacts}, we set out empirical facts, including general summary statistics and specific facts about collateral demand. In Section \ref{sec:model}, we describe our model. In Section \ref{sec:estimation}, we describe our estimation. In Section \ref{sec:results}, we set out our results. In Section \ref{sec:counterfactuals}, we describe counterfactual analysis of the importance of collateral demand. In Section \ref{sec:conclusion}, we conclude.


\subsection{Related literature}

%empirical lit on repo markets

\ach{Our primary contribution is to the empirical literature on repo markets. The early foundational papers in this literature are \cite{copeland2014repo, gorton2012securitized, krishnamurthy2014sizing} on the US repo market  and \cite{mancini2016euro} and \cite{boissel2017systemic} on European repo markets. 

Within this growing field, we contribute to three specific strands. 

The first seeks to build and estimate structural models of the repo market. Two particularly relevant papers in this literature are \cite{eisenschmidt2022monetary} and \cite{huber2023market}, who build structural models of the European and tri-party US repo market, respectively. \cite{eisenschmidt2022monetary} seek to understand the impact of market power on the pass-through of monetary policy. \cite{huber2023market} shows that market power has a material impact on spreads earned by dealers when trading with cash lenders. Our contribution is to quantify the importance of collateral demand using structural estimation.

The second aims at understanding the role of collateral demand in the repo market. When the main motivation of repo market participants shifts from funding to the trading of collateral, this can increase signs of market segmentation, as \cite{schaffner2019euro} documents for euro area money markets. A subset of the collateral demand literature focuses on the security-driven motives for trading repo and how it affects repo rates. In his foundational paper, \cite{duffie1996special} defines a 'special` as a repo rate significantly below prevailing market riskless interest rates. These can occur when competition to buy or borrow a particular bond causes buyers in the repo market to accept a lower interest rate in exchange for cash in the transaction. Recently, several empirical analyses have looked into specialness in the repo market, also against the backdrop of quantitative easing policies in major financial markets \citep{d2018scarcity,jappelli2023quantitative,roh2019repo}. Of particular relevance to our work are the findings by \cite{ballensiefen2023money}, who document that the euro money market is more segmented when the collateral motive prevails. Repo rates lent by banks with access to the deposit facility and secured by QE eligible assets are more collateral-driven and disconnected from funding-based money market rates. Our contribution is to show that collateral demand is an important and under-explored feature of repo markets. It is a key driver of repo market functioning. [X]}

\ach{The third focuses on how collateral moves through the repo market.  \cite{andolfatto2017rehypothecation}, \cite{gottardi2019theory} and \cite{infante2019liquidity}, for example, focus on rehypothecation in repo markets from a theoretical perspective. Empirical work by} \cite{singh2011velocity} and \cite{aitken2010sizable} describe the possibility of collateral rehypothecation as a lubricant to repo market functioning. Our objective is to quantify the extent to which collateral demand, in the words of \cite{singh2011velocity}, lubricates repo market functioning. \ach{Our contribution is [X].}


%market for lending assets

There is also a literature on the market for lending assets, including for the purposes of shorting. \cite{d2002market} and \cite{asquith2013market} look at depository institutions that lend equities or corporate bonds, respectively, and study what that implies for the constraints faced by arbitrageurs. Similarly, \cite{chen2022market} estimate a structural model and demonstrate how market power in the market for equities lending affects asset prices, through the effect on short sellers. We examine asset lending in the context of repo, and quantify how that relates to liquidity demand.

% why repo markets exist

Finally, there is a broad literature on why repo markets exist, given the possibility of uncollaterized lending and asset sales. Explanations include asymmetric information \cite{bigio2015endogenous} and differences of opinion \cite{geanakoplos2010leverage}. Our model is in the spirit of the latter, in that firms have different uses for cash and the bonds. Our structural model allows us to quantify such differences, and show how complementarities between funding demand and collateral demand are an important driver of repo market outcomes.






%%%% original section on lit

%%empirical lit on repo markets
%
%Our primary contribution is to the empirical literature on repo markets. \cite{copeland2014repo, gorton2012securitized, krishnamurthy2014sizing} study the role of the US repo market during the financial crisis. \cite{mancini2016euro} and \cite{boissel2017systemic} study European repo markets during the financial crisis and during the European sovereign debt crisis, respectively. \cite{huser2021repo} use a subset of the same UK data as us to track repo market performance during the Covid 19 pandemic. \cite{julliard2019drives} use a series of snapshots of similar UK data to document patterns in repo contract terms and to test different theories of why repo exists. Two particularly relevant papers in this literature are \cite{eisenschmidt2022monetary} and \cite{huber2023market}, who build structural models of the European and tri-party US repo market, respectively. \cite{eisenschmidt2022monetary} seek to understand the impact of market power on the pass-through of monetary policy. \cite{huber2023market} shows that market power has a material impact on spreads earned by dealers when trading with cash lenders.
%
%Our contribution is to show that collateral demand is an important and under-explored feature of repo markets, and to quantify its importance using structural estimation. We show how collateral demand affects market outcomes in the cross-section and inter-temporally, including during periods of financial stress.
%
%%market for lending assets
%
%There is also a literature on the market for lending assets, including for the purposes of shorting. \cite{d2002market} and \cite{asquith2013market} look at depository institutions that lend equities or corporate bonds, respectively, and study what that implies for the constraints faced by arbitrageurs. Similarly, \cite{chen2022market} estimate a structural model and demonstrate how market power in the market for equities lending affects asset prices, through the effect on short sellers. We examine asset lending in the context of repo, and quantify how that relates to liquidity demand.
%
%% rehypothecation in repo markets
%
%Our work is also related to the mostly theoretical literature on rehypothecation in repo markets, including \cite{andolfatto2017rehypothecation} and \cite{gottardi2019theory}. \cite{singh2011velocity} and \cite{aitken2010sizable} describe the possibility of collateral rehypothecation as a lubricant to repo market functioning. Our objective is to quantify the extent to which collateral demand, in the words of \cite{singh2011velocity}, lubricates repo market functioning.
%
%% why repo markets exist
%
%Finally, there is a broad literature on why repo markets exist, given the possibility of uncollaterized lending and asset sales. Explanations include asymmetric information \cite{bigio2015endogenous} and differences of opinion \cite{geanakoplos2010leverage}. Our model is in the spirit of the latter, in that firms have different uses for cash and the bonds. Our structural model allows us to quantify such differences, and show how complementarities between funding demand and collateral demand are an important driver of repo market outcomes.


%%%
%Our technical contributions are threefold: (1) we measure collateral demand more precisely, eg our structural model strips out variation in repo rates that is not due to collateral demand; (2) we measure collateral demand more completely, eg in the cross-section, which then allows us to interrogate its structure more broadly (eg effect of monetary policy on collateral demand, different types of firm); (3) we quantify the equilibrium implications of collateral demand, eg through counterfactual simulation of funding payoffs or monetary policy pass-through when collateral demand is different.